\section{Вычислительная сложность алгоритмов. Пространственная и временная сложность}
\begin{notice}
    Далее я буду безбожно копипастить лекцию Макса по соответствующей теме, которая лежит в группе.
\end{notice}
\begin{definition}
    \textbf{Классический алгоритм} - конечная последовательность операций, понятная исполнителю, строгое исполнение которой решает поставленную задачу.
\end{definition}
\begin{notice}
    Также существуют специальные виды алгоритмов (например вероятностные), для которых не всегда определены перечисленные далее св-ва, но поставленную задачу они решают.
\end{notice}

\subsection{Свойства алгоритмов}

\begin{itemize}
    \item 
        \begin{definition}
            \textbf{Дискретность} - алгоритм должен представлять процесс решения задачи как последовательное выполнение некоторых простых шагов. При этом для выполнения каждого шага алгоритма требуется конечный отрезок времени, т.е. преобразование исходных данных в результат осуществляется во времени дискретно.
        \end{definition}
    \item
        \begin{definition}
            \textbf{Детерминированность} (определённость) - в каждый момент времени следующий шаг работы однозначно определяется состоянием системы. Таким образом, алгоритм выдаёт один и тот же результат для одних и тех же исходных данных.
        \end{definition}
    \item 
        \begin{definition}
            \textbf{Понятность} - алгоритм должен включать только те команды, которые доступны исполнителю и входят в его систему команд.
        \end{definition}
    \item 
        \begin{definition}
            \textbf{Конечность} - в более узком понимании алгоритма как математической функции, при правильно заданных начальных значениях алгоритм должен завершать работу и выдавать результат за определённое число шагов. Однако довольно часто определение алгоритма не включает завершаемость за конечное время.
        \end{definition}
    \item 
        \begin{definition}
            \textbf{Универсальность} - алгоритм должен быть применим к разным наборам начальных данных.
        \end{definition}
    \item 
        \begin{definition}
            \textbf{Результативность} - завершение алгоритма определёнными задачей результатами.
        \end{definition}
\end{itemize}

\subsection{Сложность алгоритма}
\begin{definition}
    \textbf{Вычислительная сложность - } функция зависимости ресурсов, затрачиваемых некоторым алгоритмом, от размера и состояния входных данных.
\end{definition}

Основными ресурсами являются:
\begin{itemize}
    \item Процессорное время (абстракция - время) 
    \item Память (абстракция - пространство)
\end{itemize}

\begin{notice}
    Когда идёт речь о сложности алгоритма, речь идёт не о конкретных величинах (секундах), а об абстрактных, таких, как количество элементарных операций.
\end{notice}

\begin{definition}
    \textbf{Принцип trade-off} - улучшение по сложности одного ресурса за счёт ухудшения по другому.
\end{definition}

При анализе сложности алгоритма можно рассматривать сложность:
\begin{itemize}
    \item в лучшем случае
    \item в среднем случае
    \item в худшем случае
\end{itemize}

\begin{definition}
    \textbf{Гарантированной сложностью} называется сложность алгоритма в худшем случае.
\end{definition}

\begin{definition}
    \textbf{Временная сложность} - функция от размера входных данных n, равная максимальному или среднему или минимальному количеству элементарных операций, проделываемых алгоритмом для решения экземпляра задачи.
\end{definition}
\begin{definition}
    \textbf{Пространственная сложность} - аналогично определению выше, но про объём памяти.
\end{definition}